%%%%%%%%%%%%%%%%%%%%%%%%%%%%%%%%%%%%%%%%%
% License:
% The MIT License (see included LICENSE file)
%%%%%%%%%%%%%%%%%%%%%%%%%%%%%%%%%%%%%%%%%

%----------------------------------------------------------------------------------------
%	PACKAGES AND OTHER DOCUMENT CONFIGURATIONS
%----------------------------------------------------------------------------------------

\documentclass[8pt]{developercv} % Default font size, values from 8-12pt are recommended
\usepackage{graphics}
\usepackage[polish]{babel}

%----------------------------------------------------------------------------------------

\begin{document}

%----------------------------------------------------------------------------------------
%	TITLE AND CONTACT INFORMATION
% ----------------------------------------------------------------------------------------

\begin{minipage}[t]{0.45\textwidth} % 45% of the page width for name
  \vspace{-\baselineskip} % Required for vertically aligning minipages

  \colorbox{white}{{\HUGE\textcolor{black}{\textbf{\MakeUppercase{Jan}}}}} % First name

  \colorbox{white}{{\HUGE\textcolor{black}{\textbf{\MakeUppercase{Pulkowski}}}}} % Last name

  \vspace{6pt}
  \colorbox{black}{{\huge\textcolor{white}{Student Informatyki}}} % Last name
\end{minipage}
\begin{minipage}[t]{0.3\textwidth} % 30% of the page width for the row of icons
  \vspace{-\baselineskip} % Required for vertically aligning minipages

  % The first parameter is the FontAwesome icon name, the second is the box size and the third is the text
  % Other icons can be found by referring to fontawesome.pdf (supplied with the template) and using the word after \fa in the command for the icon you want
  \icon{MapMarker}{12}{Kraków/Poznań}\\
  \icon{Phone}{12}{+48 783 254 579}\\
  \icon{At}{12}{\href{mailto:pulkowski.jan.it@gmail.com}{pulkowski.jan.it@gmail.com}}\\
  \icon{Github}{12}{\href{https://github.com/dzikimlecz}{dzikimlecz}}\\
  \icon{Linkedin}{12}{\href{https://www.linkedin.com/in/pulkowski-jan}{pulkowski-jan}}\\
\end{minipage}
\hfill
\begin{minipage}[t]{0.2\textwidth}

  \vspace{-\baselineskip} % Required for vertically aligning minipages
  \includegraphics*[width=0.9\textwidth]{zdjecie}

\end{minipage}


\vspace{0.5cm}

%----------------------------------------------------------------------------------------
%	INTRODUCTION, SKILLS AND TECHNOLOGIES
%----------------------------------------------------------------------------------------

\cvsect{o mnie}

\begin{minipage}[t]{0.5\textwidth} % 50% of the page width for the introduction text
  \vspace{-\baselineskip} % Required for vertically aligning minipages
  Jestem studentem 1. roku informatyki na~Akademii Górniczo-Hutniczej w~Krakowie.
  Programowaniem i~informatyką w ogóle interesuję się od~4 lat, bardzo dobrze znam języki
  Java i~Kotlin, z~których korzystałem w~moich hobbystycznych projektach.
  Na studiach i~w~liceum korzystałem z~Pythona i~C++ do~nauki algorytmiki
  i~również dość dobrze je znam.
  Potrafię używać systemu Git i~platformy GitHub.
  Biegle korzystam z~Linuxa i~znam podstawy działania systemów Unixowych.
  Dobrze znam paradygmat programowania zorientowanego obiektowo i~standardowe praktyki programistyczne.
  % Do pracy z Javą:
  % Potrafię pisać dokumentacje w~systemie Javadoc oraz znam biblioteki Spring Boot i~Hibernate.
  Potrafię pracować w~zespole i~dobrze radzę sobie pod presją, szybko się~uczę.

\end{minipage}
\hfill % Whitespace between
\begin{minipage}[t]{0.45\textwidth} % 45% of the page for the skills bar chart
	\vspace{-\baselineskip} % Required for vertically aligning minipages
	\begin{barchart}{5.5}
		\baritem{Python}{50}
		\baritem{Kotlin}{90}
		\baritem{Java}{100}
  		\baritem{Linux}{80}
		\baritem{Git}{60}
		\baritem{C/C++}{50}
	\end{barchart}
\end{minipage}

%%%%%%%%%%%%%%%%%%%%%%%%%%%%%%%%%%%%%%%%%%%%%%%%%%%%%%%%%%%%%%%%%%%%%%
% \begin{center}													 %
% 	\bubbles{6/Intellij Idea, 3/Office, 4/Spring Boot, 5/JUnit}		 %
% \end{center}														 %
%%%%%%%%%%%%%%%%%%%%%%%%%%%%%%%%%%%%%%%%%%%%%%%%%%%%%%%%%%%%%%%%%%%%%%

%----------------------------------------------------------------------------------------
%	EDUCATION
%----------------------------------------------------------------------------------------

\cvsect{Edukacja}

\begin{entrylist}
  \entry
    {2023 -- 2027}
    {Studia inżynierskie -- Informatyka}
    {Akademia Górniczo-Hutnicza w Krakowie}
    {
      Od~2023 roku studiuję na~Akademii Górniczo-Hutniczej na~kierunku Informatyka w~programie inżynierskim osiągając bardzo dobre wyniki.
      Uczestniczyłem jako słuchacz w~konferencjach AGH Cyberkampus i~UJ Studencki Festiwal Informatyczny.
      Dotychczas w~trakcie studiów zrealizowałem kursy z~zakresu podstaw algorytmiki, programowania, matematyki, użytkowania i~podstaw systemów unixowych oraz kompetencji interpersonalnych.
    }
  \entry
    {2019 -- 2023}
    {Liceum}
    {VIII Liceum Ogólnokształcące w Poznaniu}
    {
      Uczęszczałem do~klasy o~profilu matematyczno-fizyczno-informatycznym i~zdałem maturę rozszerzoną z~informatyki, matematyki i~języka angielskiego na~bardzo wysokim poziomie.
      W~trakcie szkoły średniej uzyskałem również certyfikat z~języka angielskiego C1 Advanced (dawniej CAE) z~oceną A (poziom C2).
    }
\end{entrylist}
%----------------------------------------------------------------------------------------
%	EXPERIENCE
%----------------------------------------------------------------------------------------

\cvsect{projekty programistyczne}

\begin{entrylist}
    \entry
		{2023 -- 2024}
		{\lorem}
		{}
		{\lorem \lorem \lorem\\ \texttt{Java}\slashsep\texttt{Hibernate}}
	\entry
		{2022 -- 2023}
		{\lorem}
		{}
		{\lorem \lorem \lorem\\ \texttt{Kotlin}\slashsep\texttt{Spring Boot}\slashsep\texttt{JavaFX}}
	\entry
		{2020 -- 2021}
		{\lorem}
		{}
		{\lorem \lorem \lorem\\ \texttt{Java}\slashsep\texttt{JavaFX}}
\end{entrylist}


%----------------------------------------------------------------------------------------
%	ADDITIONAL INFORMATION
%----------------------------------------------------------------------------------------

\begin{minipage}[t]{0.3\textwidth}
  \vspace{-\baselineskip} % Required for vertically aligning minipages

  \cvsect{Języki Obce}

  \textbf{Angielski} -- zaawansowany (C2)\\
  \textbf{Niemiecki} -- komunikatywny
\end{minipage}
\hfill
\begin{minipage}[t]{0.3\textwidth}
  \vspace{-\baselineskip} % Required for vertically aligning minipages

  \cvsect{Zainteresowania}

  Moje zainteresowania obejmują ogólnie pojętą informatykę i~matematykę,
  lubię rozwiązywać wszelkiego rodzaju problemy logiczne.
  Interesuję się również polityką, literaturą piękną i~malarstwem nowoczesnym.
\end{minipage}
\hfill
\begin{minipage}[t]{0.35\textwidth}
  \vspace{-\baselineskip} % Required for vertically aligning minipages
  \cvsect{Certyfikaty}

  Lipiec 2022 -- C1 Advanced (Język angielski, poziom C2)
\end{minipage}

%----------------------------------------------------------------------------------------

\end{document}

% LocalWords:  cyberkampus
