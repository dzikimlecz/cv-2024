%%%%%%%%%%%%%%%%%%%%%%%%%%%%%%%%%%%%%%%%%
% License:
% The MIT License
%%%%%%%%%%%%%%%%%%%%%%%%%%%%%%%%%%%%%%%%%


%%%%%%%%%%%%%%%%
% KONFIGURACJA %
%%%%%%%%%%%%%%%%

\documentclass[8pt]{developercv} % Rozmiar czcionki (8-12 pt)
\usepackage{graphics}
\usepackage[polish]{babel}

%----------------------------------------------------------------------------------------

\begin{document}

%%%%%%%%%%%%%%%%%%%%%%
% NAZWISKO I KONTAKT %
%%%%%%%%%%%%%%%%%%%%%%

\begin{minipage}[t]{0.45\textwidth} % 45% szerokości na nazwisko
  \vspace{-\baselineskip} % Ustawia minipage w osi poziomej

  \colorbox{white}{{\HUGE\textcolor{black}{\textbf{\MakeUppercase{Jan}}}}} % Imie

  \colorbox{white}{{\HUGE\textcolor{black}{\textbf{\MakeUppercase{Pulkowski}}}}} % Nazwisko

  \vspace{6pt}
  \colorbox{black}{{\huge\textcolor{white}{Student Informatyki}}} % Tytuł zawodowy
\end{minipage}
\begin{minipage}[t]{0.3\textwidth}  % 45% szerokości na ikony
  \vspace{-\baselineskip} % Ustawia minipage w osi poziomej

  % \icon{nazwa}{rozmiar}{tekst}
  % lista ikon w fontawesome.pdf
  \icon{MapMarker}{12}{Kraków/Poznań}\\
  \icon{Phone}{12}{+48 783 254 579}\\
  \icon{At}{12}{\href{mailto:pulkowski.jan.it@gmail.com}{pulkowski.jan.it@gmail.com}}\\
  \icon{Github}{12}{\href{https://github.com/pulkowski-jan}{pulkowski-jan}}\\
  \icon{Linkedin}{12}{\href{https://www.linkedin.com/in/pulkowski-jan}{pulkowski-jan}}\\
\end{minipage}
\hfill
\begin{minipage}[t]{0.2\textwidth}

  \vspace{-\baselineskip} % Ustawia minipage w osi poziomej
  \includegraphics*[width=0.99\textwidth]{zdjecie}

\end{minipage}


\vspace{0.5cm}

%%%%%%%%%%%%%%%%%%%%%%%
% WSTĘP, UMIEJĘTNOŚCI %
%%%%%%%%%%%%%%%%%%%%%%%

\cvsect{o mnie}

\begin{minipage}[t]{0.5\textwidth} % 50% szerokości strony na informacje
  \vspace{-\baselineskip} % Ustawia minipage w osi poziomej
  Jestem studentem 1. roku na~kierunku Informatyka na~Akademii Górniczo-Hutniczej w~Krakowie.
  Programowaniem i~informatyką interesuję się od~4 lat, bardzo dobrze znam języki
  Java i~Kotlin, z~których korzystam w~moich hobbystycznych projektach.
  Na studiach i~w~liceum używam Pythona i~C++ do~nauki algorytmiki
  i~również dość dobrze znam te języki.
  Umiem obsłużyć Linuxa i~znam podstawy działania systemów unixowych.
  Potrafię używać systemu Git i~platformy Github.
  Dobrze znam paradygmat programowania zorientowanego obiektowo i~standardowe praktyki programistyczne.
  % Do pracy z Javą:
  % Potrafię pisać dokumentacje w~systemie Javadoc oraz znam biblioteki Spring Boot i~Hibernate.
  Potrafię pracować w~zespole i~dobrze radzę sobie pod presją czasu, szybko się~uczę.

\end{minipage}
\hfill % Przerwa
\begin{minipage}[t]{0.45\textwidth} % 45% szerokości strony na wykres umiejętności
	\vspace{-\baselineskip}
	\begin{barchart}{5.5}
		\baritem{Python}{50}
		\baritem{Kotlin}{90}
		\baritem{Java}{100}
  		\baritem{Linux}{80}
		\baritem{Git}{60}
		\baritem{C/C++}{50}
	\end{barchart}
\end{minipage}


% bąbelki jakieś idk
%%%%%%%%%%%%%%%%%%%%%%%%%%%%%%%%%%%%%%%%%%%%%%%%%%%%%%%%%%%%%%%%%%%%%%
% \begin{center}													 %
% 	\bubbles{6/Intellij Idea, 3/Office, 4/Spring Boot, 5/JUnit}		 %
% \end{center}														 %
%%%%%%%%%%%%%%%%%%%%%%%%%%%%%%%%%%%%%%%%%%%%%%%%%%%%%%%%%%%%%%%%%%%%%%

%%%%%%%%%%%%
% EDUKACJA %
%%%%%%%%%%%%

\cvsect{Edukacja}

\begin{entrylist}
  \entry
    {2023 -- 2027}
    {Studia inżynierskie -- Informatyka}
    {Akademia Górniczo-Hutnicza w Krakowie}
    {
      Od~2023 roku studiuję na~Akademii Górniczo-Hutniczej na~kierunku Informatyka w~programie inżynierskim osiągając bardzo dobre wyniki.
      Uczestniczyłem jako słuchacz w~konferencjach AGH Cyberkampus i~UJ Studencki Festiwal Informatyczny.
      Dotychczas w~trakcie studiów zrealizowałem kursy z~zakresu podstaw algorytmiki, programowania, matematyki, użytkowania i~podstaw systemów unixowych oraz kompetencji interpersonalnych.
    }
  \entry
    {2019 -- 2023}
    {Liceum}
    {VIII Liceum Ogólnokształcące w Poznaniu}
    {
      Uczęszczałem do~klasy o~profilu matematyczno-fizyczno-informatycznym i~zdałem maturę rozszerzoną z~informatyki, matematyki i~języka angielskiego na~bardzo wysokim poziomie.
      Podczas nauki w~szkole średniej uzyskałem również certyfikat z~języka angielskiego C1 Advanced (dawniej CAE) z~oceną A (poziom C2).
    }
\end{entrylist}

%%%%%%%%%%%%%%%%%%%%%%%%%%%%%%%%%%%
% PROJEKTY/DOŚWIADCZENIE ZAWODOWE %
%%%%%%%%%%%%%%%%%%%%%%%%%%%%%%%%%%%

\cvsect{Doświadczenie}

\begin{entrylist}
  \entry
    {2020 -- 2023}
    {Obsługa informatyczna gabinetu lekarskiego}
    {Indywidualna Specjalistyczna Praktyka Lekarska Anna Mania}
    {
      Realizacja statycznej strony internetowej przy użyciu HTML/CSS.
      Opracowanie programu do~układania planów zajęć studentów oraz podliczania czasu pracy prowadzących dla~osoby za~to~odpowiedzialnej.
      GUI w~JavaFX, baza danych z~lokalnym serwerem zrealizowane w~Spring Boot.
      \href{https://github.com/pulkowski-jan/timetablescheduler}{
        \textcolor{blue}{\underline{GitHub}}
      }
    }
\end{entrylist}

\cvsect{projekty programistyczne}

\begin{entrylist}
  \entry
    {2023 -- \ldots}
    {LibGtfsKt}
    {\icon{Github}{5}{\href{https://github.com/pulkowski-jan/libgtfskt-core}{\underline{Kod}}}}
    {
      Biblioteka umożliwiająca pobieranie informacji o~rozkładach jazdy komunikacji miejskiej operatorów obsługujących system GTFS.
      Docelowo będzie możliwe zapisywanie i~manipulowanie danymi w~bazach relacyjnych.
      \texttt{Kotlin}\slashsep\texttt{Hibernate}
    }
  \entry
    {2020 -- 2021}
    {Discord chess}
    {\icon{Github}{5}{\href{https://github.com/pulkowski-jan/Discord-chess}{\underline{Kod}}}}
    {
      W~pełni funkcjonalny bot do~komunikatora Discord symulujący grę w~szachy 2 graczy.
      Gra jest na~bieżąco wizualizowana poprzez generowane grafiki.
      Symulacja gry w~szachy jest realizowana poprzez samodzielną bibliotekę, również mojego autorstwa.
      \\\texttt{Java}\slashsep\texttt{JUnit 5}
    }
\end{entrylist}


%%%%%%%%%%%%%%%%%%%%%%%%
% DODATKOWE INFORMACJE %
%%%%%%%%%%%%%%%%%%%%%%%%

\begin{minipage}[t]{0.3\textwidth}
  \vspace{-\baselineskip}

  \cvsect{Języki Obce}

  \textbf{Angielski} -- zaawansowany (C2)\\
  \textbf{Niemiecki} -- komunikatywny
\end{minipage}
\hfill
\begin{minipage}[t]{0.3\textwidth}
  \vspace{-\baselineskip}

  \cvsect{Zainteresowania}

  Moje zainteresowania obejmują informatykę i~matematykę,
  lubię rozwiązywać wszelkiego rodzaju problemy logiczne.
  Interesuję się również polityką, literaturą piękną i~malarstwem nowoczesnym.
\end{minipage}
\hfill
\begin{minipage}[t]{0.35\textwidth}
  \vspace{-\baselineskip}
  \cvsect{Certyfikaty}

  Lipiec 2022 -- C1 Advanced (Język angielski, poziom C2)
\end{minipage}

%----------------------------------------------------------------------------------------

\end{document}

% LocalWords:  cyberkampus
